% Options for packages loaded elsewhere
\PassOptionsToPackage{unicode}{hyperref}
\PassOptionsToPackage{hyphens}{url}
\PassOptionsToPackage{dvipsnames,svgnames,x11names}{xcolor}
%
\documentclass[
  letterpaper,
  DIV=11,
  numbers=noendperiod]{scrreprt}

\usepackage{amsmath,amssymb}
\usepackage{iftex}
\ifPDFTeX
  \usepackage[T1]{fontenc}
  \usepackage[utf8]{inputenc}
  \usepackage{textcomp} % provide euro and other symbols
\else % if luatex or xetex
  \usepackage{unicode-math}
  \defaultfontfeatures{Scale=MatchLowercase}
  \defaultfontfeatures[\rmfamily]{Ligatures=TeX,Scale=1}
\fi
\usepackage{lmodern}
\ifPDFTeX\else  
    % xetex/luatex font selection
\fi
% Use upquote if available, for straight quotes in verbatim environments
\IfFileExists{upquote.sty}{\usepackage{upquote}}{}
\IfFileExists{microtype.sty}{% use microtype if available
  \usepackage[]{microtype}
  \UseMicrotypeSet[protrusion]{basicmath} % disable protrusion for tt fonts
}{}
\makeatletter
\@ifundefined{KOMAClassName}{% if non-KOMA class
  \IfFileExists{parskip.sty}{%
    \usepackage{parskip}
  }{% else
    \setlength{\parindent}{0pt}
    \setlength{\parskip}{6pt plus 2pt minus 1pt}}
}{% if KOMA class
  \KOMAoptions{parskip=half}}
\makeatother
\usepackage{xcolor}
\setlength{\emergencystretch}{3em} % prevent overfull lines
\setcounter{secnumdepth}{5}
% Make \paragraph and \subparagraph free-standing
\makeatletter
\ifx\paragraph\undefined\else
  \let\oldparagraph\paragraph
  \renewcommand{\paragraph}{
    \@ifstar
      \xxxParagraphStar
      \xxxParagraphNoStar
  }
  \newcommand{\xxxParagraphStar}[1]{\oldparagraph*{#1}\mbox{}}
  \newcommand{\xxxParagraphNoStar}[1]{\oldparagraph{#1}\mbox{}}
\fi
\ifx\subparagraph\undefined\else
  \let\oldsubparagraph\subparagraph
  \renewcommand{\subparagraph}{
    \@ifstar
      \xxxSubParagraphStar
      \xxxSubParagraphNoStar
  }
  \newcommand{\xxxSubParagraphStar}[1]{\oldsubparagraph*{#1}\mbox{}}
  \newcommand{\xxxSubParagraphNoStar}[1]{\oldsubparagraph{#1}\mbox{}}
\fi
\makeatother


\providecommand{\tightlist}{%
  \setlength{\itemsep}{0pt}\setlength{\parskip}{0pt}}\usepackage{longtable,booktabs,array}
\usepackage{calc} % for calculating minipage widths
% Correct order of tables after \paragraph or \subparagraph
\usepackage{etoolbox}
\makeatletter
\patchcmd\longtable{\par}{\if@noskipsec\mbox{}\fi\par}{}{}
\makeatother
% Allow footnotes in longtable head/foot
\IfFileExists{footnotehyper.sty}{\usepackage{footnotehyper}}{\usepackage{footnote}}
\makesavenoteenv{longtable}
\usepackage{graphicx}
\makeatletter
\newsavebox\pandoc@box
\newcommand*\pandocbounded[1]{% scales image to fit in text height/width
  \sbox\pandoc@box{#1}%
  \Gscale@div\@tempa{\textheight}{\dimexpr\ht\pandoc@box+\dp\pandoc@box\relax}%
  \Gscale@div\@tempb{\linewidth}{\wd\pandoc@box}%
  \ifdim\@tempb\p@<\@tempa\p@\let\@tempa\@tempb\fi% select the smaller of both
  \ifdim\@tempa\p@<\p@\scalebox{\@tempa}{\usebox\pandoc@box}%
  \else\usebox{\pandoc@box}%
  \fi%
}
% Set default figure placement to htbp
\def\fps@figure{htbp}
\makeatother

\KOMAoption{captions}{tableheading}
\makeatletter
\@addtoreset{chapter}{part}
\makeatother
\makeatletter
\@ifpackageloaded{bookmark}{}{\usepackage{bookmark}}
\makeatother
\makeatletter
\@ifpackageloaded{caption}{}{\usepackage{caption}}
\AtBeginDocument{%
\ifdefined\contentsname
  \renewcommand*\contentsname{Table of contents}
\else
  \newcommand\contentsname{Table of contents}
\fi
\ifdefined\listfigurename
  \renewcommand*\listfigurename{List of Figures}
\else
  \newcommand\listfigurename{List of Figures}
\fi
\ifdefined\listtablename
  \renewcommand*\listtablename{List of Tables}
\else
  \newcommand\listtablename{List of Tables}
\fi
\ifdefined\figurename
  \renewcommand*\figurename{Figure}
\else
  \newcommand\figurename{Figure}
\fi
\ifdefined\tablename
  \renewcommand*\tablename{Table}
\else
  \newcommand\tablename{Table}
\fi
}
\@ifpackageloaded{float}{}{\usepackage{float}}
\floatstyle{ruled}
\@ifundefined{c@chapter}{\newfloat{codelisting}{h}{lop}}{\newfloat{codelisting}{h}{lop}[chapter]}
\floatname{codelisting}{Listing}
\newcommand*\listoflistings{\listof{codelisting}{List of Listings}}
\makeatother
\makeatletter
\makeatother
\makeatletter
\@ifpackageloaded{caption}{}{\usepackage{caption}}
\@ifpackageloaded{subcaption}{}{\usepackage{subcaption}}
\makeatother

\usepackage{bookmark}

\IfFileExists{xurl.sty}{\usepackage{xurl}}{} % add URL line breaks if available
\urlstyle{same} % disable monospaced font for URLs
\hypersetup{
  pdftitle={Mount Everest Climber Services},
  pdfauthor={Jiahao Tong},
  colorlinks=true,
  linkcolor={blue},
  filecolor={Maroon},
  citecolor={Blue},
  urlcolor={Blue},
  pdfcreator={LaTeX via pandoc}}


\title{Mount Everest Climber Services}
\author{\href{https://tommytong21.github.io/casa0025groupwork/}{Jiahao
Tong}}
\date{2025-10-04}

\begin{document}
\maketitle

\renewcommand*\contentsname{Table of contents}
{
\hypersetup{linkcolor=}
\setcounter{tocdepth}{2}
\tableofcontents
}

\bookmarksetup{startatroot}

\chapter*{Module Overview}\label{module-overview}
\addcontentsline{toc}{chapter}{Module Overview}

\markboth{Module Overview}{Module Overview}

Geospatial analytics and dashboards are in very high remand among
policymakers, NGOs, IGOs, and the private sector. Deploying these
systems often requires handling data that exceeds the computational and
storage capabilities of personal machines. This module will teach
students how to harness and critically interrogate large quantities of
geospatial data using cloud computing services, and how to design and
build an interactive online application that communicates geospatial
insights to wider audiences.

In line with this objective, the module is divided into two sections. In
the first, database concepts and techniques are introduced, providing
the students with the skills required to manipulate and derive meaning
from organised datasets. SQL syntax will be taught in depth at this
stage, with a strong emphasis on practical application. This will allow
students to learn state of the art methods for handling large vector
datasets.

The second section of the course focuses on the handling of large raster
datasets. As geospatial datasets---particularly satellite imagery
collections---increase in size, researchers are increasingly relying on
cloud computing platforms such as Google Earth Engine (GEE) to analyze
vast quantities of data. Despite the fact that it was only released in
2015, the number of geospatial journal articles using Google Earth
Engine has outpaced every other major geospatial analysis software,
including ArcGIS, Python, and R in just five years. Weeks 6-9 will be
co-taught with CASA0023 Remote Sensing.

The module therefore spans a full, cloud-based geospatial workflow: from
importing and analysing geospatial data, to building and presenting
interactive data visualisations. Students will gain proficiency in
working with and interrogating large spatial data sets while working
towards an interactive group project that will develop their portfolio.

\section*{What is SQL?}\label{what-is-sql}
\addcontentsline{toc}{section}{What is SQL?}

\markright{What is SQL?}

SQL (Structured Query Language) is a programming language used to
communicate with databases. It is the standard language for relational
database management systems. SQL statements are used to perform tasks
such as update data on a database, or retrieve data from a database.
Some common relational database management systems that use SQL are:
Oracle, Sybase, Microsoft SQL Server, Access, Ingres, etc. Although most
database systems use SQL, most of them also have their own additional
proprietary extensions that are usually only used on their system.
However, the standard SQL commands such as ``Select'', ``Insert'',
``Update'', ``Delete'', ``Create'', and ``Drop'' can be used to
accomplish almost everything that one needs to do with a database.

The first five weeks of this module will focus on working with large
vector datasets. We will use SQL to query and manipulate data stored in
a PostgreSQL database. PostgreSQL is a free and open-source relational
database management system emphasizing extensibility and SQL compliance.
It is the most advanced open-source database system widely used for GIS
applications. We will also work with DuckDB, a new, open-source,
in-process SQL OLAP database management system. DuckDB is designed to be
used as an embedded database library, providing C/C++, Python, R, Java,
and Go bindings. It has a built-in SQL engine with support for
transactions, a powerful query optimizer, and a columnar storage engine.

\section*{What is Google Earth
Engine?}\label{what-is-google-earth-engine}
\addcontentsline{toc}{section}{What is Google Earth Engine?}

\markright{What is Google Earth Engine?}

As geospatial datasets---particularly satellite imagery
collections---increase in size, researchers are increasingly relying on
cloud computing platforms such as Google Earth Engine (GEE) to analyze
vast quantities of data.

GEE is free and allows users to write open-source code that can be run
by others in one click, thereby yielding fully reproducible results.
These features have put GEE on the cutting edge of scientific research.
The following plot visualizes the number of journal articles conducted
using different geospatial analysis software platforms:

\pandocbounded{\includegraphics[keepaspectratio]{./images/WoS Articles.png}}

Despite only being released in 2015, the number of geospatial journal
articles using Google Earth Engine (shown in red above) has outpaced
every other major geospatial analysis software, including ArcGIS,
Python, and R in just five years. GEE applications have been developed
and used to present interactive geospatial data visualizations by NGOs,
Universities, the United Nations, and the European Commission. By
storing and running computations on google servers, GEE is far more
accessible to those who don't have significant local computational
resources; all you need is an internet connection.

\section*{Table of Contents}\label{table-of-contents}
\addcontentsline{toc}{section}{Table of Contents}

\markright{Table of Contents}

\begin{enumerate}
\def\labelenumi{\Alph{enumi})}
\tightlist
\item
  \textbf{SQL}

  \begin{itemize}
  \tightlist
  \item
    Five initial weeks exporing spatial database systems including
    PostgreSQL and DuckDB.

    \begin{itemize}
    \tightlist
    \item
      \href{intro.qmd}{Introduction}
    \item
      \href{test.qmd}{Test}
    \end{itemize}
  \end{itemize}
\item
  \textbf{Google Earth Engine}

  \begin{itemize}
  \tightlist
  \item
    Five weeks on Google Earth Engine, a cloud-based platform for
    geospatial analysis.

    \begin{itemize}
    \tightlist
    \item
      \href{services.qmd}{services}
    \end{itemize}
  \end{itemize}
\item
  \textbf{Conclusion}

  \begin{itemize}
  \tightlist
  \item
    Five weeks on Google Earth Engine, a cloud-based platform for
    geospatial analysis.

    \begin{itemize}
    \tightlist
    \item
      \href{conclusion.qmd}{conclusion}
    \end{itemize}
  \end{itemize}
\end{enumerate}

\part{A. introduction}

\chapter{}\label{section}

\chapter{我的网页}\label{ux6211ux7684ux7f51ux9875}

\chapter{This is a test script for the `Climber'
project.}\label{this-is-a-test-script-for-the-climber-project.}

\part{B. services}

\chapter{}\label{section-1}

\part{C. conclusion}

\chapter{}\label{section-2}

\bookmarksetup{startatroot}

\chapter{我的网页}\label{ux6211ux7684ux7f51ux9875-1}

\bookmarksetup{startatroot}

\chapter{This is a test script for the `Climber'
project.}\label{this-is-a-test-script-for-the-climber-project.-1}




\end{document}
